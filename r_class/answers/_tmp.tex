\documentclass[]{article}
\usepackage{lmodern}
\usepackage{amssymb,amsmath}
\usepackage{ifxetex,ifluatex}
\usepackage{fixltx2e} % provides \textsubscript
\ifnum 0\ifxetex 1\fi\ifluatex 1\fi=0 % if pdftex
  \usepackage[T1]{fontenc}
  \usepackage[utf8]{inputenc}
\else % if luatex or xelatex
  \ifxetex
    \usepackage{mathspec}
  \else
    \usepackage{fontspec}
  \fi
  \defaultfontfeatures{Ligatures=TeX,Scale=MatchLowercase}
\fi
% use upquote if available, for straight quotes in verbatim environments
\IfFileExists{upquote.sty}{\usepackage{upquote}}{}
% use microtype if available
\IfFileExists{microtype.sty}{%
\usepackage{microtype}
\UseMicrotypeSet[protrusion]{basicmath} % disable protrusion for tt fonts
}{}
\usepackage[margin=1.0in]{geometry}
\usepackage{hyperref}
\hypersetup{unicode=true,
            pdftitle={Chapter 16 Answers},
            pdfborder={0 0 0},
            breaklinks=true}
\urlstyle{same}  % don't use monospace font for urls
\usepackage{color}
\usepackage{fancyvrb}
\newcommand{\VerbBar}{|}
\newcommand{\VERB}{\Verb[commandchars=\\\{\}]}
\DefineVerbatimEnvironment{Highlighting}{Verbatim}{commandchars=\\\{\}}
% Add ',fontsize=\small' for more characters per line
\usepackage{framed}
\definecolor{shadecolor}{RGB}{248,248,248}
\newenvironment{Shaded}{\begin{snugshade}}{\end{snugshade}}
\newcommand{\KeywordTok}[1]{\textcolor[rgb]{0.13,0.29,0.53}{\textbf{#1}}}
\newcommand{\DataTypeTok}[1]{\textcolor[rgb]{0.13,0.29,0.53}{#1}}
\newcommand{\DecValTok}[1]{\textcolor[rgb]{0.00,0.00,0.81}{#1}}
\newcommand{\BaseNTok}[1]{\textcolor[rgb]{0.00,0.00,0.81}{#1}}
\newcommand{\FloatTok}[1]{\textcolor[rgb]{0.00,0.00,0.81}{#1}}
\newcommand{\ConstantTok}[1]{\textcolor[rgb]{0.00,0.00,0.00}{#1}}
\newcommand{\CharTok}[1]{\textcolor[rgb]{0.31,0.60,0.02}{#1}}
\newcommand{\SpecialCharTok}[1]{\textcolor[rgb]{0.00,0.00,0.00}{#1}}
\newcommand{\StringTok}[1]{\textcolor[rgb]{0.31,0.60,0.02}{#1}}
\newcommand{\VerbatimStringTok}[1]{\textcolor[rgb]{0.31,0.60,0.02}{#1}}
\newcommand{\SpecialStringTok}[1]{\textcolor[rgb]{0.31,0.60,0.02}{#1}}
\newcommand{\ImportTok}[1]{#1}
\newcommand{\CommentTok}[1]{\textcolor[rgb]{0.56,0.35,0.01}{\textit{#1}}}
\newcommand{\DocumentationTok}[1]{\textcolor[rgb]{0.56,0.35,0.01}{\textbf{\textit{#1}}}}
\newcommand{\AnnotationTok}[1]{\textcolor[rgb]{0.56,0.35,0.01}{\textbf{\textit{#1}}}}
\newcommand{\CommentVarTok}[1]{\textcolor[rgb]{0.56,0.35,0.01}{\textbf{\textit{#1}}}}
\newcommand{\OtherTok}[1]{\textcolor[rgb]{0.56,0.35,0.01}{#1}}
\newcommand{\FunctionTok}[1]{\textcolor[rgb]{0.00,0.00,0.00}{#1}}
\newcommand{\VariableTok}[1]{\textcolor[rgb]{0.00,0.00,0.00}{#1}}
\newcommand{\ControlFlowTok}[1]{\textcolor[rgb]{0.13,0.29,0.53}{\textbf{#1}}}
\newcommand{\OperatorTok}[1]{\textcolor[rgb]{0.81,0.36,0.00}{\textbf{#1}}}
\newcommand{\BuiltInTok}[1]{#1}
\newcommand{\ExtensionTok}[1]{#1}
\newcommand{\PreprocessorTok}[1]{\textcolor[rgb]{0.56,0.35,0.01}{\textit{#1}}}
\newcommand{\AttributeTok}[1]{\textcolor[rgb]{0.77,0.63,0.00}{#1}}
\newcommand{\RegionMarkerTok}[1]{#1}
\newcommand{\InformationTok}[1]{\textcolor[rgb]{0.56,0.35,0.01}{\textbf{\textit{#1}}}}
\newcommand{\WarningTok}[1]{\textcolor[rgb]{0.56,0.35,0.01}{\textbf{\textit{#1}}}}
\newcommand{\AlertTok}[1]{\textcolor[rgb]{0.94,0.16,0.16}{#1}}
\newcommand{\ErrorTok}[1]{\textcolor[rgb]{0.64,0.00,0.00}{\textbf{#1}}}
\newcommand{\NormalTok}[1]{#1}
\usepackage{graphicx,grffile}
\makeatletter
\def\maxwidth{\ifdim\Gin@nat@width>\linewidth\linewidth\else\Gin@nat@width\fi}
\def\maxheight{\ifdim\Gin@nat@height>\textheight\textheight\else\Gin@nat@height\fi}
\makeatother
% Scale images if necessary, so that they will not overflow the page
% margins by default, and it is still possible to overwrite the defaults
% using explicit options in \includegraphics[width, height, ...]{}
\setkeys{Gin}{width=\maxwidth,height=\maxheight,keepaspectratio}
\IfFileExists{parskip.sty}{%
\usepackage{parskip}
}{% else
\setlength{\parindent}{0pt}
\setlength{\parskip}{6pt plus 2pt minus 1pt}
}
\setlength{\emergencystretch}{3em}  % prevent overfull lines
\providecommand{\tightlist}{%
  \setlength{\itemsep}{0pt}\setlength{\parskip}{0pt}}
\setcounter{secnumdepth}{0}
% Redefines (sub)paragraphs to behave more like sections
\ifx\paragraph\undefined\else
\let\oldparagraph\paragraph
\renewcommand{\paragraph}[1]{\oldparagraph{#1}\mbox{}}
\fi
\ifx\subparagraph\undefined\else
\let\oldsubparagraph\subparagraph
\renewcommand{\subparagraph}[1]{\oldsubparagraph{#1}\mbox{}}
\fi

%%% Use protect on footnotes to avoid problems with footnotes in titles
\let\rmarkdownfootnote\footnote%
\def\footnote{\protect\rmarkdownfootnote}

%%% Change title format to be more compact
\usepackage{titling}

% Create subtitle command for use in maketitle
\providecommand{\subtitle}[1]{
  \posttitle{
    \begin{center}\large#1\end{center}
    }
}

\setlength{\droptitle}{-2em}

  \title{Chapter 16 Answers}
    \pretitle{\vspace{\droptitle}\centering\huge}
  \posttitle{\par}
    \author{}
    \preauthor{}\postauthor{}
    \date{}
    \predate{}\postdate{}
  

\begin{document}
\maketitle

\pagenumbering{gobble}

\textbf{Highlights}

\begin{itemize}
\tightlist
\item
  Review these functions

  \begin{itemize}
  \tightlist
  \item
    data.frame
  \item
    head
  \item
    tail
  \item
    summary
  \item
    transform
  \item
    subset
  \end{itemize}
\end{itemize}

\textbf{16.1} What is returned by the following R commands? (Waking
hours from wikipedia.)

\begin{Shaded}
\begin{Highlighting}[]
\NormalTok{creatures =}\StringTok{ }\KeywordTok{c}\NormalTok{(}\StringTok{"dog"}\NormalTok{,}\StringTok{"cat"}\NormalTok{,}\StringTok{"armadillo"}\NormalTok{,}\StringTok{"human"}\NormalTok{)}
\NormalTok{friendly =}\StringTok{ }\KeywordTok{c}\NormalTok{(}\OtherTok{TRUE}\NormalTok{,}\OtherTok{TRUE}\NormalTok{,}\OtherTok{FALSE}\NormalTok{,}\OtherTok{TRUE}\NormalTok{)}
\NormalTok{diet =}\StringTok{ }\KeywordTok{c}\NormalTok{(}\StringTok{"cats"}\NormalTok{,}\StringTok{"mice"}\NormalTok{,}\StringTok{"termites"}\NormalTok{,}\StringTok{"Twinkies(tm)"}\NormalTok{)}
\NormalTok{waking.hours =}\StringTok{ }\KeywordTok{c}\NormalTok{(}\FloatTok{13.9}\NormalTok{, }\FloatTok{11.5}\NormalTok{, }\FloatTok{5.9}\NormalTok{, }\FloatTok{16.0}\NormalTok{)}
\NormalTok{creature.data =}\StringTok{ }\KeywordTok{data.frame}\NormalTok{ (friendly, diet, waking.hours, }\DataTypeTok{row.names=}\NormalTok{creatures)}
\NormalTok{creature.data}
\end{Highlighting}
\end{Shaded}

\begin{verbatim}
##           friendly         diet waking.hours
## dog           TRUE         cats         13.9
## cat           TRUE         mice         11.5
## armadillo    FALSE     termites          5.9
## human         TRUE Twinkies(tm)         16.0
\end{verbatim}

\begin{Shaded}
\begin{Highlighting}[]
\NormalTok{creature.data[[}\DecValTok{2}\NormalTok{]]}
\end{Highlighting}
\end{Shaded}

\begin{verbatim}
## [1] cats         mice         termites     Twinkies(tm)
## Levels: cats mice termites Twinkies(tm)
\end{verbatim}

\begin{Shaded}
\begin{Highlighting}[]
\NormalTok{creature.data[}\DecValTok{2}\NormalTok{]}
\end{Highlighting}
\end{Shaded}

\begin{verbatim}
##                   diet
## dog               cats
## cat               mice
## armadillo     termites
## human     Twinkies(tm)
\end{verbatim}

\begin{Shaded}
\begin{Highlighting}[]
\NormalTok{creature.data[}\DecValTok{2}\NormalTok{,}\DecValTok{2}\NormalTok{]}
\end{Highlighting}
\end{Shaded}

\begin{verbatim}
## [1] mice
## Levels: cats mice termites Twinkies(tm)
\end{verbatim}

\begin{Shaded}
\begin{Highlighting}[]
\NormalTok{creature.data}\OperatorTok{$}\NormalTok{waking.hours}\OperatorTok{>}\DecValTok{12} \OperatorTok{&}\StringTok{ }\OperatorTok{!}\NormalTok{creature.data}\OperatorTok{$}\NormalTok{friendly}
\end{Highlighting}
\end{Shaded}

\begin{verbatim}
## [1] FALSE FALSE FALSE FALSE
\end{verbatim}

\begin{Shaded}
\begin{Highlighting}[]
\NormalTok{creatures[creature.data}\OperatorTok{$}\NormalTok{waking.hours }\OperatorTok{<}\StringTok{ }\DecValTok{12} \OperatorTok{&}\StringTok{ }\OperatorTok{!}\NormalTok{creature.data}\OperatorTok{$}\NormalTok{friendly]}
\end{Highlighting}
\end{Shaded}

\begin{verbatim}
## [1] "armadillo"
\end{verbatim}

\textbf{16.2} Write a single R command that alphabetizes the rows of the
data frame \texttt{creature.data} from Exercise 16.1 by creature name.

\begin{Shaded}
\begin{Highlighting}[]
\NormalTok{creature.data[}\KeywordTok{sort}\NormalTok{(creatures),]}
\end{Highlighting}
\end{Shaded}

\begin{verbatim}
##           friendly         diet waking.hours
## armadillo    FALSE     termites          5.9
## cat           TRUE         mice         11.5
## dog           TRUE         cats         13.9
## human         TRUE Twinkies(tm)         16.0
\end{verbatim}

\begin{Shaded}
\begin{Highlighting}[]
\CommentTok{# also (perhaps the separate object creatures was not created)}
\NormalTok{ creature.data[}\KeywordTok{sort}\NormalTok{(}\KeywordTok{row.names}\NormalTok{(creature.data)),]}
\end{Highlighting}
\end{Shaded}

\begin{verbatim}
##           friendly         diet waking.hours
## armadillo    FALSE     termites          5.9
## cat           TRUE         mice         11.5
## dog           TRUE         cats         13.9
## human         TRUE Twinkies(tm)         16.0
\end{verbatim}

\begin{Shaded}
\begin{Highlighting}[]
\CommentTok{# for fun:}
\KeywordTok{summary}\NormalTok{(creature.data)}
\end{Highlighting}
\end{Shaded}

\begin{verbatim}
##   friendly                 diet    waking.hours  
##  Mode :logical   cats        :1   Min.   : 5.90  
##  FALSE:1         mice        :1   1st Qu.:10.10  
##  TRUE :3         termites    :1   Median :12.70  
##                  Twinkies(tm):1   Mean   :11.82  
##                                   3rd Qu.:14.43  
##                                   Max.   :16.00
\end{verbatim}

\textbf{16.3} Use the R \texttt{subset} function to create a data frame
consisting of just the creature name and diet associated with friendly
creatures who are awake more than 12 hours a day from the data drame
\texttt{creature.data} from Exercise 16.1.

\begin{Shaded}
\begin{Highlighting}[]
\KeywordTok{subset}\NormalTok{(creature.data,friendly }\OperatorTok{&}\StringTok{ }\NormalTok{(waking.hours }\OperatorTok{>}\StringTok{ }\DecValTok{12}\NormalTok{),}\KeywordTok{c}\NormalTok{(diet))}
\end{Highlighting}
\end{Shaded}

\begin{verbatim}
##               diet
## dog           cats
## human Twinkies(tm)
\end{verbatim}

\textbf{16.4} Consider the data frame `\texttt{creature.data} from
Exercise 16.1.

\begin{itemize}
\tightlist
\item
  Extract the waking hours for a dog using two different R commands
\end{itemize}

\begin{Shaded}
\begin{Highlighting}[]
\NormalTok{creature.data[}\StringTok{"dog"}\NormalTok{,}\StringTok{"waking.hours"}\NormalTok{]}
\end{Highlighting}
\end{Shaded}

\begin{verbatim}
## [1] 13.9
\end{verbatim}

\begin{Shaded}
\begin{Highlighting}[]
\CommentTok{# subset extracts a new data frame,}
\CommentTok{# adding the [1,1] indexes extracts the cell value}
\KeywordTok{subset}\NormalTok{(creature.data,}\StringTok{"dog"} \OperatorTok{==}\StringTok{ }\KeywordTok{row.names}\NormalTok{(creature.data),}\KeywordTok{c}\NormalTok{(waking.hours))[}\DecValTok{1}\NormalTok{,}\DecValTok{1}\NormalTok{]}
\end{Highlighting}
\end{Shaded}

\begin{verbatim}
## [1] 13.9
\end{verbatim}

\begin{itemize}
\tightlist
\item
  Extract the waking hours for all creatures using two different R
  commands
\end{itemize}

\begin{Shaded}
\begin{Highlighting}[]
\NormalTok{creature.data}\OperatorTok{$}\NormalTok{waking.hours}
\end{Highlighting}
\end{Shaded}

\begin{verbatim}
## [1] 13.9 11.5  5.9 16.0
\end{verbatim}

\begin{Shaded}
\begin{Highlighting}[]
\NormalTok{creature.data[[}\StringTok{"waking.hours"}\NormalTok{]]}
\end{Highlighting}
\end{Shaded}

\begin{verbatim}
## [1] 13.9 11.5  5.9 16.0
\end{verbatim}

\begin{Shaded}
\begin{Highlighting}[]
\KeywordTok{subset}\NormalTok{(creature.data,}\OtherTok{TRUE}\NormalTok{,}\KeywordTok{c}\NormalTok{(waking.hours))[,}\DecValTok{1}\NormalTok{]}
\end{Highlighting}
\end{Shaded}

\begin{verbatim}
## [1] 13.9 11.5  5.9 16.0
\end{verbatim}

\begin{itemize}
\tightlist
\item
  Create a data frame that consists only of the rows for dogs and
  armadillos using two different R commands
\end{itemize}

\begin{Shaded}
\begin{Highlighting}[]
\KeywordTok{subset}\NormalTok{(creature.data,}\StringTok{"dog"} \OperatorTok{==}\StringTok{ }\KeywordTok{row.names}\NormalTok{(creature.data) }\OperatorTok{|}\StringTok{ "armadillo"} \OperatorTok{==}\StringTok{ }\KeywordTok{row.names}\NormalTok{(creature.data),}\KeywordTok{colnames}\NormalTok{(creature.data))}
\end{Highlighting}
\end{Shaded}

\begin{verbatim}
##           friendly     diet waking.hours
## dog           TRUE     cats         13.9
## armadillo    FALSE termites          5.9
\end{verbatim}

\begin{Shaded}
\begin{Highlighting}[]
\KeywordTok{subset}\NormalTok{(creature.data,}\StringTok{"dog"} \OperatorTok{==}\StringTok{ }\KeywordTok{row.names}\NormalTok{(creature.data) }\OperatorTok{|}\StringTok{ "armadillo"} \OperatorTok{==}\StringTok{ }\KeywordTok{row.names}\NormalTok{(creature.data),)}
\end{Highlighting}
\end{Shaded}

\begin{verbatim}
##           friendly     diet waking.hours
## dog           TRUE     cats         13.9
## armadillo    FALSE termites          5.9
\end{verbatim}

\begin{Shaded}
\begin{Highlighting}[]
\KeywordTok{subset}\NormalTok{(creature.data,(waking.hours }\OperatorTok{<}\StringTok{ }\DecValTok{16} \OperatorTok{&}\StringTok{ }\NormalTok{waking.hours}\OperatorTok{>}\DecValTok{12}\NormalTok{) }\OperatorTok{|}\StringTok{ }\NormalTok{waking.hours}\OperatorTok{<}\DecValTok{6}\NormalTok{,)}
\end{Highlighting}
\end{Shaded}

\begin{verbatim}
##           friendly     diet waking.hours
## dog           TRUE     cats         13.9
## armadillo    FALSE termites          5.9
\end{verbatim}

\begin{itemize}
\tightlist
\item
  Create a data frame that consists only of friendly creatures
\end{itemize}

\begin{Shaded}
\begin{Highlighting}[]
\KeywordTok{subset}\NormalTok{(creature.data,friendly,)}
\end{Highlighting}
\end{Shaded}

\begin{verbatim}
##       friendly         diet waking.hours
## dog       TRUE         cats         13.9
## cat       TRUE         mice         11.5
## human     TRUE Twinkies(tm)         16.0
\end{verbatim}

\textbf{16.5} Consider a data frame named \texttt{a} with 4 rows and 3
columns populated with positive integers and -1 where -1 denotes a
missing value. Write an R command that replaces each -1 with NA.

\begin{Shaded}
\begin{Highlighting}[]
\NormalTok{a =}\StringTok{ }\KeywordTok{data.frame}\NormalTok{(}\DataTypeTok{col1=}\DecValTok{1}\OperatorTok{:}\DecValTok{4}\NormalTok{,}\DataTypeTok{col2=}\KeywordTok{seq}\NormalTok{(}\DecValTok{10}\NormalTok{,}\DecValTok{40}\NormalTok{,}\DataTypeTok{by=}\DecValTok{10}\NormalTok{),}\DataTypeTok{col3=}\KeywordTok{seq}\NormalTok{(}\DecValTok{100}\NormalTok{,}\DecValTok{400}\NormalTok{,}\DataTypeTok{by=}\DecValTok{100}\NormalTok{))}
\NormalTok{a}
\end{Highlighting}
\end{Shaded}

\begin{verbatim}
##   col1 col2 col3
## 1    1   10  100
## 2    2   20  200
## 3    3   30  300
## 4    4   40  400
\end{verbatim}

\begin{Shaded}
\begin{Highlighting}[]
\NormalTok{a[}\DecValTok{1}\NormalTok{,}\DecValTok{1}\NormalTok{] =}\StringTok{ }\NormalTok{a[}\DecValTok{2}\NormalTok{,}\DecValTok{2}\NormalTok{] =}\StringTok{ }\NormalTok{a[}\DecValTok{3}\NormalTok{,}\DecValTok{3}\NormalTok{] =}\StringTok{ }\NormalTok{a}\OperatorTok{$}\NormalTok{col2[}\DecValTok{4}\NormalTok{] =}\StringTok{ }\DecValTok{-1}
\NormalTok{a}
\end{Highlighting}
\end{Shaded}

\begin{verbatim}
##   col1 col2 col3
## 1   -1   10  100
## 2    2   -1  200
## 3    3   30   -1
## 4    4   -1  400
\end{verbatim}

\begin{Shaded}
\begin{Highlighting}[]
\NormalTok{a[a}\OperatorTok{==-}\DecValTok{1}\NormalTok{] <-}\StringTok{ }\OtherTok{NA}
\NormalTok{a}
\end{Highlighting}
\end{Shaded}

\begin{verbatim}
##   col1 col2 col3
## 1   NA   10  100
## 2    2   NA  200
## 3    3   30   NA
## 4    4   NA  400
\end{verbatim}

\textbf{16.6} Create the objects \texttt{v}, \texttt{m}, \texttt{a},
\texttt{l}, and \texttt{d} as a vector, matrix, array, list, and
data.frame. Apply the functions \texttt{class}, \texttt{typeof}, and
\texttt{mode}to each of the objects and summarize the results in a
table.

\begin{Shaded}
\begin{Highlighting}[]
\NormalTok{v =}\StringTok{ }\DecValTok{1}\OperatorTok{:}\DecValTok{12}
\NormalTok{m =}\StringTok{ }\KeywordTok{matrix}\NormalTok{(v,}\DecValTok{3}\NormalTok{,}\DecValTok{4}\NormalTok{)}
\NormalTok{a =}\StringTok{ }\KeywordTok{array}\NormalTok{(v,}\KeywordTok{c}\NormalTok{(}\DecValTok{2}\NormalTok{,}\DecValTok{2}\NormalTok{,}\DecValTok{3}\NormalTok{))}
\NormalTok{l =}\StringTok{ }\KeywordTok{list}\NormalTok{(v,m,a)}
\NormalTok{d =}\StringTok{ }\KeywordTok{subset}\NormalTok{(creature.data,friendly,)}

\NormalTok{st =}\StringTok{ }\KeywordTok{data.frame}\NormalTok{(}
    \DataTypeTok{class  =}    \KeywordTok{c}\NormalTok{(  }\KeywordTok{class}\NormalTok{(v),   }\KeywordTok{class}\NormalTok{(m),   }\KeywordTok{class}\NormalTok{(a),   }\KeywordTok{class}\NormalTok{(l),   }\KeywordTok{class}\NormalTok{(d)), }
    \DataTypeTok{typeof =}    \KeywordTok{c}\NormalTok{(  }\KeywordTok{typeof}\NormalTok{(v),  }\KeywordTok{typeof}\NormalTok{(m),  }\KeywordTok{typeof}\NormalTok{(a),  }\KeywordTok{typeof}\NormalTok{(l),  }\KeywordTok{typeof}\NormalTok{(d)),}
    \DataTypeTok{mode   =}    \KeywordTok{c}\NormalTok{(  }\KeywordTok{mode}\NormalTok{(v),    }\KeywordTok{mode}\NormalTok{(m),    }\KeywordTok{mode}\NormalTok{(a),    }\KeywordTok{mode}\NormalTok{(l),    }\KeywordTok{mode}\NormalTok{(d)),}
    \DataTypeTok{row.names =} \KeywordTok{c}\NormalTok{(  }\StringTok{"v"}\NormalTok{,        }\StringTok{"m"}\NormalTok{,        }\StringTok{"a"}\NormalTok{,        }\StringTok{"l"}\NormalTok{,        }\StringTok{"d"}\NormalTok{)}
\NormalTok{)}
\NormalTok{st}
\end{Highlighting}
\end{Shaded}

\begin{verbatim}
##        class  typeof    mode
## v    integer integer numeric
## m     matrix integer numeric
## a      array integer numeric
## l       list    list    list
## d data.frame    list    list
\end{verbatim}

\textbf{16.7} Create a data frame named \texttt{w} that consists of the
following three named columns:

\begin{itemize}
\tightlist
\item
  \texttt{x}, the first four positve integers,
\item
  \texttt{y}, the abbreviations of the first four months,
\item
  \texttt{z}, the first names of the Beatles.
\end{itemize}

\begin{Shaded}
\begin{Highlighting}[]
\NormalTok{w =}\StringTok{ }\KeywordTok{data.frame}\NormalTok{( }\DataTypeTok{x =} \DecValTok{1}\OperatorTok{:}\DecValTok{4}\NormalTok{,}
                \DataTypeTok{y =}\NormalTok{ month.abb[}\DecValTok{1}\OperatorTok{:}\DecValTok{4}\NormalTok{],}
                \DataTypeTok{z =} \KeywordTok{sort}\NormalTok{(}\KeywordTok{c}\NormalTok{(}\StringTok{"John"}\NormalTok{,}\StringTok{"Paul"}\NormalTok{,}\StringTok{"Ringo"}\NormalTok{,}\StringTok{"George"}\NormalTok{))}
\NormalTok{)}
\NormalTok{w}
\end{Highlighting}
\end{Shaded}

\begin{verbatim}
##   x   y      z
## 1 1 Jan George
## 2 2 Feb   John
## 3 3 Mar   Paul
## 4 4 Apr  Ringo
\end{verbatim}

Remove the second column of \texttt{w} in the following two fashions:

\begin{itemize}
\tightlist
\item
  set \texttt{w\$y} to \texttt{NULL}, that is, \texttt{w\$y\ =\ NULL},
\item
  subset the columns to keep, that is, \texttt{w{[}c("x","z"){]}}.
\end{itemize}

\begin{Shaded}
\begin{Highlighting}[]
\NormalTok{w_ =}\StringTok{ }\NormalTok{w}
\NormalTok{w_}\OperatorTok{$}\NormalTok{y =}\StringTok{ }\OtherTok{NULL}

\NormalTok{w__ =}\StringTok{ }\NormalTok{w}
\NormalTok{w__[}\KeywordTok{c}\NormalTok{(}\StringTok{"x"}\NormalTok{,}\StringTok{"z"}\NormalTok{)]}
\end{Highlighting}
\end{Shaded}

\begin{verbatim}
##   x      z
## 1 1 George
## 2 2   John
## 3 3   Paul
## 4 4  Ringo
\end{verbatim}

Show that the first technique alters the data frame \texttt{w}but the
second technique does not alter the data frame \texttt{w}.

\begin{Shaded}
\begin{Highlighting}[]
\CommentTok{# the first method was an assignment operation, resulting in a change}
\NormalTok{w_}
\end{Highlighting}
\end{Shaded}

\begin{verbatim}
##   x      z
## 1 1 George
## 2 2   John
## 3 3   Paul
## 4 4  Ringo
\end{verbatim}

\begin{Shaded}
\begin{Highlighting}[]
\CommentTok{# the second method was an extraction operation, with no permanent damage done to the data frame}
\NormalTok{w__}
\end{Highlighting}
\end{Shaded}

\begin{verbatim}
##   x   y      z
## 1 1 Jan George
## 2 2 Feb   John
## 3 3 Mar   Paul
## 4 4 Apr  Ringo
\end{verbatim}

\textbf{Exercises taken from indicated chapter of ``Learning Base R'',
by Lawrence M Leemis, ISBN 978-0-9829174-8-0}


\end{document}
